\section{Introduction}\label{sec:introduction}

Modern railways represent a major form of transport with an ever-growing user base, as trains are flexible in terms of travelling distance (they can be used for local, regional and long-distance services) and capacity (as they are modular by nature). Furthermore, train transportation is usually the greenest transportation options for both goods and people.

Despite this, railways are confronted with the increase of operational costs and a fierce competition from other modes of transport. Many users demand more reliability in train operations: a long delay, a cancelled train, a missed connection can easily decrease the perceived quality of service and turn away potential customers.

Most of the events that negatively affect train operations (broadly called \emph{conflicts}) happen when, for some reason, there is a difference between the nominal and the actual service. The causes of such events are usually divided into \emph{disturbances} and \emph{disruptions} (\citet{Cacchiani2014}). The former are small perturbations of the system that are handled by network operators by momentarily changing the timetable. The results of disturbances are usually minor, such as one or more delayed trains, or a platform change at a station. Disruptions, on the other hand, are major incidents that not only alter the nominal timetable, but also require changes in rolling stock and crews. The outcome of a disruption could include major delays, train cancellations, and long reroutings. Disturbances clearly happen much more often than disruptions and their impact is not to be underestimated: a train that is delayed just a few minutes can make a user miss an important connection and increase their travel time by hours. In this paper we consider both disturbances and disruptions in a unified way, by defining an algorithmics approach to handle the conflicts they cause.

Increasing systemwide reliability is crucial at every phase of the planning process. It starts at the strategic and tactical levels (budget allocation for maintenance, timetable robustness, etc.), but once at the operational level, it is almost impossible to avoid that day-to-day activities be disturbed by many kinds of unforeseen events.

When such an event occurs, it is the job of the \emph{dispatcher} to restore the system in a working state. The job of dispatchers has been traditionally done by hand, based exclusively on their experience and practice. It was not until recent years that computer algorithms were developed with the aim of aiding the dispatchers in making the best decision that resolves the critical situation and minimises deviances from the nominal timetable.

In this paper we present such an algorithm, developed to solve the Train Rescheduling Problem (TRP): given a nominal timetable which has become infeasible because of one or more \emph{conflicts} that have arisen, we are asked to produce a new conflict-free timetable that is as close as possible to the nominal one. Or, in case it is not possible to produce a conflict-free timetable, we need to warn the dispatcher about this and provide a timetable with the least possible number of conflicts.

Conflicts are all those situations that either can't physically happen (e.g., two trains occupying the same segment of track at the same time) or that can potentially compromise the safety of operations in the network (e.g., two trains running too close to each other in the same direction).

The algorithm presented in this paper is the result of a long lasting collaboration with Alstom, initiated by the company in 2012 with the aim of redesigning the optimisation algorithms incorporated in its Train Management System ICONIS. To this end, Alstom involved three important Italian research groups in specific research projects investigating various optimisation problems arising in the real time conflict resolution. As a result of such initial wide research effort, the team formed by Optit, an accredited spinoff of the University of Bologna, and the Department of Electrical, Electronic and Information Engineering of the University of Bologna, was selected to produce an innovative real-time conflict solution algorithm capable of taking into account the characteristics and constraints of practical applications which has been developed and industrialised during 2013, and extensively tested by Alstom in real-world contexts. Recently, the new algorithm has been fully integrated in ICONIS and will be deployed at various international Alstom customers.

The paper is structured as follows. In the next section we give an overview of how a railway system works, how it can be affected by disturbances and what it means to reschedule a train. In \Cref{sec:literature_review} we review the existing literature on the TRP, based on the classification schema given by \citet{Cacchiani2014}. In \Cref{sec:model} we give a mathematical description of a railway network, of train timetables and of the relationship between them. We present an heuristic algorithm for the solution of the TRP in \Cref{sec:solution_algorithm}. We then describe the instances used and provide computational results in \Cref{sec:computational_results}. Finally, we draw conclusions and propose further research paths in \Cref{sec:conclusions}.
