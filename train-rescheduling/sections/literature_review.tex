\section{Literature Review}\label{sec:literature_review}

Conflict resolution in train applications, often known as the \emph{train dispatching problem} (see, e.g., \citet{meng2014simultaneous}), is widely studied in the literature, and research contributions can be classified in several ways.

A first possible subdivision may take into account the level of detail used in modelling the physical resources composing the train network. In this respect, the main distinction was usually between \emph{microscopic} and \emph{macroscopic} modelling approaches. A microscopic approach would represent every element of the rail infrastructure in detail (individual tracks, platforms, etc.). In such a model every network element can be assigned to only one train at a time, thus leading to \emph{explicit} capacity requirements on the resources. A typical macroscopic model, on the other hand, would disregard any fine-grained segmentation of the tracks, thus leading to \emph{cumulative} capacity requirements, since each network element could represent several physical resources. In the literature, such models are also known, respectively, as \emph{single-track} and \emph{$N$-track} models (see, e.g., \citet{Tornquist2007}). This distinction, however, is often blurry, and several authors adopted a mixed approach, by considering so-called \emph{mesoscopic} models, in which the modelling detail is not specified a priori. Here, network elements can represent either low level infrastructure, such as specific tracks or platforms, or aggregate one, such as entire stations or $N$-track segments.

Another widely used subdivision takes into account the type of conflict resolution actions available to the decision makers. These include the application of \emph{retiming}, \emph{reordering}, \emph{retracking}, and \emph{rerouting} of trains (\citet{meng2014simultaneous}). Such actions involve, respectively: the adjustment of speeds and stopping times; modifying the order in which trains occupy platforms or track segments; small and large changes in the path followed by trains in the network.

In their recent survey, \citet{Cacchiani2014} also adopted a classification scheme which mainly takes into account the type of conflict to be managed by the model. More precisely, the authors distinguished between \emph{disturbances} and \emph{disruptions} and analysed the literature classifying models and solution approaches based on this viewpoint. The reader is also referred to \citet{Tornquist2007,meng2014simultaneous} for additional literature analyses and classification. Other classification schemes proposed in recent surveys mainly focus on the solution methodology adopted (see \citet{fang2015survey}) or on dynamic and stochastic components related to on-line rescheduling (see \citet{corman2015review}). Finally, we direct the interested reader to the recent book of \citet{hansen2014railway} for a comprehensive analysis of many aspects of railway timetabling and operations, including train rescheduling.

Many works which employ a more microscopic approach revolve around the concept of \emph{alternative graph}, introduced by \citet{Mascis2002} for the no-wait job shop scheduling. The problem of assigning a train to a track segment for a certain period of time, in fact, can be seen as a job shop scheduling problem where track segment are machines and the assignment of a train to a segment is an operation. Additional constraints, such as set-up times and no-wait constraints, are used to model specific characteristics of the problem. The alternative graph formulation was widely used to develop solution approaches to various rescheduling problems (see, e.g., \citet{DAriano2007}) such as the ROMA tool (see, e.g., \citet{DAriano2008a,DAriano2008b,Corman2009,Corman2010a,Corman2010b,Corman2011,Corman2012,DAriano2009}).
\citet{corman2016integrating} integrates fast heuristics for this model, which provide a primal solution, with a network-flow relaxation, providing a dual bound.

Other approaches, which use alternative solution paradigms, have also been explored. \citet{Rodriguez2007} solved conflicts using constraint programming techniques and using the job shop model with additional constraints. \citet{Meng2011} propose a stochastic programming model is used to reschedule trains on a single-track line, so that the new schedule is robust. \citet{pellegrini2014optimal} solve a real-time traffic management problem using a pure Mixed-Integer Programming (MIP) model which represents a small section of a railway network with fine granularity. \citet{sama2016ant} use an ant-colony optimisation metaheuristic to select the best routing alternative for each train in a real-time setting. A simulation-based approach for train dispatching was proposed by \citet{li2008efficient}. \citet{mu2011scheduling} employs fuzzy optimisation techniques to reschedule trains after a low-probability disruption occurs. Finally, 

Several authors tried to bridge the gap between fine-grained and more aggregate representations by using different techniques. For example, \citet{Lamorgese2013,Lamorgese2015} propose an iterative macro- and microscopic approach, in which the \emph{line traffic control} problem takes care of the macroscopic constraints (trains meeting at stations, stations' capacities respected) and acts as a master problem. The \emph{station traffic control} considers instead detailed constraints at the station level and acts as a subproblem to generate cuts for the master problem, in a way analogous to Benders decomposition. Other mesoscopic approaches have been based on MIP formulations: \citet{tornquist2007n} used an exact model for rescheduling on $N$-track networks; \citet{Tornquist2012} used a MIP-based greedy heuristic starting from the same model; this model was further extended by \citet{Acuna2011}, who also consider intermediate stops and bidirectional tracks.

While minimising the total delay is a sensible choice, in recent years the focus of rescheduling techniques has been shifting towards a more passenger-oriented point of view, which aims to minimise the travellers' delay. In this spirit, \citet{Schobel2007} solved the delay management problem, consisting in deciding which connections between trains should be maintained, even when this would mean to introduce some delay on certain trains that would have to wait for others. The work has been expanded in \citet{Schobel2009,Schachtebeck2010,Dollevoet2014}, while \citet{Kanai2011} propose a combined optimisation/simulation algorithm that allows to track additional performance indicators other than total passenger delay. On the other hand, concering the scheduling of freight trains, \citet{mu2011scheduling} recently proposed effective heuristic approaches based on decomposition.
