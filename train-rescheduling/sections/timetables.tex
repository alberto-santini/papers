\section{Timetables and conflicts}\label{sec:timetables}

Nominal timetables are the crucial part of any railway systems. They describe in detail the trip of each train, from its departure to its arrival station, including all the intermediate stations where the train stops or passes by. This includes not only those parts of the trip where the train operates passenger service, but also all the movements necessary to perform service and maintenance, e.g., rolling stock relocation, cleaning, technical service.

Every arrival and departure is scheduled at specific time slots, which are calculated in advance by taking into account physical properties (e.g., track curvature and gradient, maximum allowed speed, train length) and interaction among trains. Clearly two trains can't occupy the same portion of tracks at the same time, but other constraints usually have to be respected. For example trains have to respect \emph{headway times}, i.e., a minimum amount of time must be left as a buffer between trains travelling in the same direction. Another example are \emph{dwell times} at platforms, which are needed to board and alight passengers.

Timetables can be \emph{periodic} or \emph{aperiodic}. Periodic timetables repeat themselves at certain time intervals (e.g., every second hour and every hour during peak times). Although such timetables are usually appreciated by customers, as they are easy to memorise and use, they are difficult to implement in a competitive market where many train operators are likely to request access to the same resources at the same time. For this reason, trains are often scheduled in aperiodic timetables. The name \emph{aperiodic} is slightly misleading, since these timetables are repeated day after day so, strictly speaking, they have a period of one day.

Timetables are implemented by assigning \emph{tasks} to \emph{rolling stock} and \emph{crews}. When it comes to passenger transportation, rolling stock are usually composed of one or more locomotives and many passenger cars; or, in case of multiple unit (MU) trains (MU trains are those composed by one or more similar self-propelled train cars), by one or more MUs. A crew includes a train driver and one or more train guards. Finally, a task represents a complete trip of the rolling stock and the crew from the train origin to its destination. The set of tasks carried out by rolling stock and crews in a day is called a \emph{shift}, or duty.

Since in most countries the railway infrastructure is operated by a different actor than the trains, the timetables are usually created and managed by an \emph{infrastructure manager}, who tries to accommodate the requests of train operators as much as possible, while abiding to safety rules and other operational constraints. Once the timetables are set up, train operators will assign rolling stock and crews to the corresponding tasks.

During real-life operations a train can easily deviate from its nominal timetable: extra time might be needed at a station to board and alight passengers, weather conditions might force the driver to slow down in certain parts of the route, etc. These are examples of \emph{primary} delays. A delayed train, in fact, could interfere with the operations of other trains, in turn delaying them (\emph{secondary} delays) and many delays can end up knocking on from one train to another.

As already mentioned in \Cref{sec:introduction}, in this work we consider disturbances and disruptions (introduced in \Cref{sec:introduction}) under a unified umbrella. A detailed list of the conflicts we consider is given in \Cref{ssec:conflicts}. The corrective actions that our algorithm will suggest are limited to \emph{retiming}, \emph{respeeding}, and \emph{rerouting} trains, collectively named \emph{rescheduling}. Retiming consists in changing the durations of train stops at stations. Respeeding changes the times trains enter and leave different parts of the network (i.e., changing their speed). Finally, rerouting consists in assigning a train a new path in the network.

Several criteria can be considered when rescheduling a set of trains. For example, we may want to minimise the deviance from the nominal timetable, or the total delay, or the number of broken connections, etc. In our work, we present a general way of modelling events in the network, that is able to take into account all of these criteria (and many more).
