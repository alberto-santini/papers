\section{The ALNS Framework}\label{sec:alns}

ALNS was introduced by \autocite{ropke2006adaptive} and extends the LNS metaheuristic first proposed by \autocite{Shaw98}.
In the LNS, we consider a neighbourhood which is \rev{implicitly} defined by the sequential application of a \emph{destroy} and a \emph{repair} method.
A destroy method turns a feasible solution into an incomplete solution, by destroying parts of it;
a repair method then takes an incomplete solution and turns it into a feasible solution.
In ALNS, we consider a collection of destroy and repair methods.
A neighbourhood is implicitly defined for each possible pair of destroy and repair methods, assuming that any repair method is able to reconstruct a solution from an incomplete solution created by any destroy method.

\rev{For example, a destroy method for a Vehicle Routing Problem could remove a certain number of customers from their respective routes.}
\rev{A repair method could then try to re-insert the missing customers, trying all feasible positions and choosing the one that minimises the total cost.}

Some element of randomness is commonly introduced in the process.
This element is usually included in the destroy method, by randomising the choice of which parts of the solution to destroy.
In most implementations, the repair methods aim to, myopically, obtain the best possible solution starting from an incomplete solution;
however, it is also possible to introduce some stochastic element in the repair methods.
At each iteration, the destroy and repair methods are chosen based on their past performance, reflected by a score:
the methods are picked with a roulette-wheel selection, where the probabilities are directly proportional to the scores.
Initially all methods are assigned the same score.

\begin{algorithm}[H]
  \caption{General Framework}\nllabel{alg:general_fw}
  \SetKwInOut{Input}{Input}

  \Input{Initial solution: $x_0$}
  \Input{Initial acceptance parameters}
  \Input{Initial destroy/repair scores}

  $x = x_0$\tcc*{Initialise current solution}
  $x^* = x_0$\tcc*{Initialise best solution}
  $i=1$\tcc*{Iteration count}

  \While{$i \leq K$}{
    Choose a destroy method $d$\;
    Choose a repair method $r$\;
    $x' \leftarrow r(d(x))$\;
    \If{Accept new solution $x'$}{ \nllabel{ln:accept}
      $x = x'$
    }

    \If{$f(x) < f(x^*)$}{
      $x^* = x$
    }

    Update(Destroy/repair scores)\; \nllabel{ln:update_sco}
    Update(Acceptance parameters)\; \nllabel{ln:update_acc}
    $i = i + 1$
  }

  \Return $x^*$
\end{algorithm}

A synthetic formulation of the ALNS algorithm is given in \Cref{alg:general_fw}.
Once the destroy and repair methods are chosen, a new solution is produced.
The algorithm then has to decide whether or not to replace the \emph{current} solution with the one newly produced --- thus \emph{accepting} or \emph{rejecting} the new solution.
The criterion used to decide whether or not the \rev{new solution} is accepted is therefore called the \emph{acceptance criterion}.
The criterion itself can base the acceptance decision on some internal state, which can vary during the course of the solution process.
For example, a Simulated Annealing (SA) criterion has been the most popular choice when implementing ALNS:
in the case of SA, the varying state is represented by the temperature, which starts at a high value and exponentially decreases during the execution of the algorithm.

When the \rev{new solution} improves on the global best solution, the scores of the corresponding destroy and repair methods are increased by a relatively large value;
otherwise, if the new solution is accepted, their scores are increased by a relatively smaller value;
otherwise, if the new solution is not accepted, their scores are decreased.

In our implementation, the solution process ends when we reach a predetermined number of iterations.
Other criteria that have been used include a hard time limit, and a predetermined number of consecutive iterations without improvement.

\secrev{
The LNS and ALNS has similarities to the method known as iterated greedy, first proposed by \autocite{RuSt07}.
That first implementation of iterated greedy only uses a single destroy method (based on random removal) and a single repair method.
As opposed to ALNS, it explicitly includes an optional local search improvement phase.
\autocite{Gl00} points out that alternating constructive and destructive phases corresponds to a one-sided strategic oscillation.
Strategic oscillation refers to a process where a search is strategically driven to go back and forth between the set of feasible solutions and a set of infeasible solutions \parencite{GlLa97}.
In this context, destroy methods create infeasible solutions, and the repair methods drive the resulting infeasible solution towards a feasible solution.
Finally, the ruin and recreate heuristic, proposed in \parencite{Schrimpf00} is also very similar to LNS and ALNS in the sense that it improves an initial solution by repeatedly destroying (ruining) and repairing (recreating) the current solution.
}