\section{Test Problems}\label{sec:problems}

To evaluate the different acceptance criteria, we consider ALNS implementations for three different combinatorial optimisation problems, as presented below.

\subsection{Capacitated Vehicle Routing Problem}

In the Capacitated Vehicle Routing Problem (CVRP) we have to deliver goods from a depot to a set of customers, using an unlimited fleet of identical vehicles.
Each customer demands a certain quantity of goods and the vehicles have a limited capacity.
Our task is to construct routes starting and ending at the depot that minimise the total travel distance and that obey the capacity of the vehicles.
We assume that travel distances are symmetric in the sense that the distance from A to B is the same as the distance from B to A.
The problem can be modelled on a directed graph $G = (N, A)$ where the node set is $N = \{ 0, \ldots, n\}$ and node $0$ represents the depot, while nodes $C = \{ 1, \ldots, n\}$ represent the customers.
Each customer $i \in C$ has an associated demand $q_i \geq 0$ and the vehicles all have the same capacity $Q \geq \max_{i \in C} q_i$.

In the literature on heuristics for the CVRP, researchers have typically also considered instances that include a distance or duration limit for each route.
In the standard benchmark instances, customers have a service time and for each route the sum of service times plus distance driven has to be less than or equal to a threshold $L$.
For more information the reader is referred to \autocite{irnich2014family} and \autocite{laporte2014heuristics}.

\subsection{Capacitated Minimum Spanning Tree Problem}

In the (symmetric) Capacitated Minimum Spanning Tree (CMST) we have to construct a spanning tree subject to a capacity constraint.
The problem is defined on a undirected graph $G=(N,E)$ where $N$ is the node set and $E$ are the edges.
For each edge $e\in E$ we are given an associated cost $c_e \geq 0$.
In the node set $N=\left\{ 0,\ldots,n \right\}$, node $0$ is the root node.
The remaining nodes $i \in N \setminus \left\{ 0 \right\}$ are associated with a demand $d_i \geq 0$ and we are given a maximum demand or capacity $Q$.
Removing node 0 from any spanning tree results in the tree splitting into one or more connected components.
In the CMST, the solution has to satisfy the property that the sum of the demands of each component (or sub-tree) is less than or equal to $Q$ (capacity constraints).
We seek the spanning tree that minimizes the sum of edge costs while satisfying capacity constraints.
For more information on this problem, see \autocite{uchoa2008robust}.

\subsection{Quadratic Assignment Problem}

\rev{The Linear Assignment Problem (LAP) aims at assigning $n$ facilities to $n$ locations.}
\rev{The assignment of facility $i$ to location $j$ incurs in a cost $c_{ij}$.}
\rev{The objective of the LAP is to minimise the total cost, while each facility is assigned to exactly one location, and each location receives exactly one facility.}
\rev{The Quadratic Assignment Problem (QAP) is an extension of the LAP, where the costs are associated to pairs of assignments:}
\rev{we are given $n^4$ costs, with $c_{ijkl}$ corresponding to the simultaneous assignment of facility $i$ to location $j$, and facility $k$ to location $l$.}
\rev{A cost can be thought of as $c_{ijkl} = f_{ik} d_{jl}$ where $f_{ik}$ is the flow to be sent from facility $i$ to facility $k$, and $d_{jl}$ is the distance between location $j$ and location $l$.}
\rev{More information about this problem can be found in the seminal paper by \autocite{lawler1963quadratic}.}
\rev{For works on metaheuristics for the QAP see, e.g., \autocite{stutzle2006iterated,james2009multistart}.}
