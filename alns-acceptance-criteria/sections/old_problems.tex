\section{Problems considered}

\subsection{Capacitated Vehicle Routing Problem}

In the Capacitated Vehicle Routing Problem (CVRP) we have to deliver goods from a depot to a set of customers, using an unlimited fleet of identical vehicles. Each customer demands a certain quantity of goods and the vehicles have a limited capacity. Our task is to construct routes starting and ending at the depot that minimise the total travel distance and that obey the capacity of the vehicles. Usually it is assumed that travel distances are symmetric in the sense that the distance from A to B is the same as the distance from B to A. In case this assumption does not hold the problem is denoted the asymmetric vehicle routing problem. We consider the symmetric case since this by far is most studied in the literature.

In the literature on heuristic for the CVRP researchers have typically also considered instances that include a distance or duration limit for each route. In the standard benchmark instances customers have a service time and for each route the sum of service times plus distance driven has to be less than or equal to a threshold $L$.

\citet{laporte2014heuristics}
\citet{IrToVi14}

%The problem can be modeled on a undirected graph $G = (N, E)$ where the node set is $N = \{ 0, \ldots, n\}$ and node $0$ represents the depot, while nodes $C = \{ 1, \ldots, n\}$ represent the customers. Each customer $i \in C$ has an associated demand $q_i \geq 0$ and the vehicles all have the same capacity $Q \geq \max_{i \in C} q_i$.

\subsection{Capacitated Minimum Spanning Tree Problem}
In the (symmetric) Capacitated Minimum Spanning Tree (CMST) we have to construct a spanning tree subject to a capacity constraint. The problem is defined on a undirected graph $G=(N,E)$ where $N$ is the node set and $E$ are the edges. For each edge $e\in E$ we are given an associated cost $c_e$. In the node set $N={0,\ldots,n}$ node 0 is the root node. The remaining nodes $i\in N\setminus {0}$ are associated with a demand $d_i \geq 0$ and we are given a maximum demand or capacity $Q$. Removing node 0 from any spanning tree will result in the tree splitting into 1 or more connected components. In the CMST the solution has to satisfy that the sum of the demand of each component (or sub-tree) is less than or equal to $Q$. We 

\citet{uchoa2008robust}